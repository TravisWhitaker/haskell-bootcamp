\documentclass{beamer}

\usepackage{listings}
\usepackage{hyperref}

% Make Beamer less ugly:
\usefonttheme{serif}
\setbeamercolor{title}{fg=black}
\setbeamercolor{titlelike}{fg=black}
\setbeamercolor{frametitles}{fg=black}
\setbeamertemplate{itemize items}{\color{black}\textbullet}

\frenchspacing

\lstset{basicstyle=\small}
\lstset{language=C}

\begin{document}

\title{Haskell Bootcamp \\ Module 1: Setup, Tools, and Resources}
\author{Travis Whitaker}
\date{}

\maketitle

\begin{frame}
\frametitle{Welcome to Haskell!}
I hope you're ready to:
\begin{itemize}
\item Think about programs from a new and valuable perspective.
\item Quickly create programs that are short, fast, and correct on the first
      try.
\item Leverage modern advancements (1980's onward) to structure your software
      into generic, reusable, verifyable pieces.
\item Leave behind memory safety issues, buggy concurrency, and unhandled
      runtime failures.
\end{itemize}
\end{frame}

\begin{frame}
\frametitle{Non-Goals}
\begin{itemize}
\item Covering every nook and cranny of Haskell's syntax and features.
\item Learning Category Theory, Homotopy Type Theory, or other theoretical
      nonsense.
\item Examining the lowest-level parts of the language, e.g. manually managing
      memory, exactly how programs are compiled to machine code, etc.
\item Decoding FP snob esotera, e.g. what does \texttt{((.) . (.) . (.))} do?
\item See also the ``Zero Bullshit Haskell'' project.
\end{itemize}
\end{frame}

\begin{frame}
\frametitle{Resources: Haskell Programming from First Principles}
\begin{itemize}
\item \url{http://haskellbook.com/}
\item Best Haskell book currently available.
\item Focuses on intuition-building.
\item Loads of good coding exercises.
\item I've used this to successfully bootstrap interns in under two weeks.
\end{itemize}
\end{frame}

\begin{frame}
\frametitle{Resources: Real World Haskell}
\begin{itemize}
\item \url{http://book.realworldhaskell.org/read/}
\item Old, but gold.
\item Exposition of core language functionality is still relevant.
\item Chapters exploring specific libraries are mostly out of date.
\end{itemize}
\end{frame}

\begin{frame}
\frametitle{Resources: Learn You a Haskell for Great Good}
\begin{itemize}
\item \url{http://learnyouahaskell.com/}
\item Good early exposition of syntax and semantics.
\item Quickly moves on to advanced features.
\item No content related to building whole working programs.
\end{itemize}
\end{frame}

\begin{frame}
\frametitle{Resources: Parallel and Concurrent Programming in
            Haskell
           }
\begin{itemize}
\item \url{https://www.oreilly.com/library/view/parallel-and-concurrent/9781449335939/}
\item Intermediate level material, assumes some basic Haskell knowledge.
\item Excellent exposition of concurrent Haskell.
\item Written by the dude who wrote the GHC runtime system.
\item Includes loads of useful patterns, even if you aren't programming in
      Haskell.
\end{itemize}
\end{frame}

\begin{frame}
\frametitle{Resources: Hackage}
\begin{itemize}
\item \url{https://hackage.haskell.org}
\item This is Haskell's public package reposity.
\item Think PyPI, Rust Crates, CPAN, etc.
\item Really good search; whatever you need is probably already there.
\item Generated documentation for all packages.
\end{itemize}
\end{frame}

\end{document}
